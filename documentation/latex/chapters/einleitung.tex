\chapter{Einleitung}

Im Zuge der Digitalisierung werden alle Bereiche des täglichen Lebens auf neue, digitale Systeme umgestellt. So wird beispielsweise im Bereich des Wohnens auf Technologien gesetzt, die über das Internet steuerbar sind. Damit einhergehend wird oft der Begriff des Internet of Things (IoT) genannt, zu dem auch der Bereich Smart Home gezählt werden kann.\\
Diese Dokumentation beschäftigt sich mit dem Entwurf und der Umsetzung einer Smart Home- Architektur und der Implementierung zweier Apps, die für die Steuerung der Komponenten in diesem Netzwerk genutzt werden sollen. Die erste App soll eine LED-Lichterkette in verschiedenen Modi ansteuern können. Die zweite Anwendung soll eine Multitimer-App darstellen, die als wichtiger Helfer im Haushalt und insbesondere beim Kochen agiert. Dabei ist diese Timer-App ebensfalls an das Smart Home-Netzwerk angebunden und kann einzelne Komponenten des Smart Homes steuern. \\
Das Ergebnis dieser Arbeit ist ein prototypisch implementiertes Netzwerk, in das die beiden Apps eingepflegt werden und die IoT-Geräte angesteuert werden können.