\newglossaryentry{ecm}{
	type=\acronymtype,
	name={ECM},
	description={Enterprise Content Management},
	first={Enterprise Content Management (ECM)},
	%see=[Glossar:]{ecmg}
}

%\newglossaryentry{ecmg}{
%	name={Enterprise Content Management System},
%	description={Enterprise Content Management Systeme dienen zur Erfassung, Verwaltung, Speicherung, Bewahrung und Ausgabe von Informationen}
%}


\newglossaryentry{xml}{
	type=\acronymtype,
	name={XML},
	description={eXtensible Markup Language},
	%see=[Glossar:]{xmlg}
}

%\newglossaryentry{xmlg}{
%	name={eXtensible Markup Language},
%	description={XMl ist eine vom W3C verabschiedete Auszeichnungssprache, mit der sich hierarchische Strukturen darstellen lassen. Die erstellten Dokumente sind vom Menschen lesbar und bieten ein Format, welches mit dem Computer verarbeitet werden kann.}
%}

\newglossaryentry{Dokument}{
	name={Dokument},
	plural={Dokumente},
	description={Laut der ImageMaster Terminologie besitzt ein Dokument eine spezifische ID und ist einem \gls{Dokumententyp} zugeordnet. Einem Dokument können mehrere Dateien zugeordnet werden}
}

\newglossaryentry{Dokumententyp}{
	name={Dokumententyp},
	plural={Dokumententypen},
	description={Ein Dokumententyp beinhaltet Attribute nach denen später gesucht werden kann. \textbf{Beispiel:} Eine Rechnung mit den Attributen Rechnungsdatum, Kundennummer und Rechnungsbetrag.}
}

\newglossaryentry{AIIM}{
	type=\acronymtype,
	name={AIIM},
	description={Association for Information and Image Management},
	first={\textit{Association for Information and Image Management} (AIIM)},
	%see=[Glossar:]{aiimg}
}

\newglossaryentry{aiimg}{
	name={Association for Information and Image Management},
	first={\textit{Association for Information and Image Management} (AIIM)},
	description={Der AIIM ist eine Organisation, welche sich das Ziel gesetzt hat, Benutzern zu helfen die Herausforderungen in der Dokumentenverwaltung zu verstehen.}
}

\newglossaryentry{Revisionssicherheit}{
	name={Revisionssicherheit},
	description={Eine Revisionssichere Archivierung von Dokumenten bedeutet, dass diese nicht mit neueren Versionen überschrieben werden. Möchte man eine aktualisierte Version eines Dokuments abspeichern, bleiben alle älteren Versionen erhalten und es kann weiterhin auf sie zugegriffen werden.}
}

\newglossaryentry{Java EE}{
	type=\acronymtype,
	name={Java EE},
	description={Java Enterprise Edition},
	first={\textit{Java Enterprise Edition} (Java EE)},
%see=[Glossar:]{javaee}
}

\newglossaryentry{javaee}{
	name={Java EE},
	description={Java Enterprise Edition ist eine Spezifikation für die Softwarearchitektur auf Java basierenter Anwendungen, insbesondere Web-Anwendungen. Die Spezifikation definiert Dienste und Softwarekomponenten, sowie deren Schnittstellen über die sie mit anderen Komponenten kommunizieren können. Anhand dieser Spezifikation lassen sich mehrschichtige Anwendungen aufbauen und auf mehrere Systeme vertielen.}
}

\newglossaryentry{SOAP} {
	type=\acronymtype,
	name={SOAP},
	description={Simple Object Access Protocol},
	first={\textit{Simple Object Access Protocol} (SOAP)},
	%see=[Glossar:]{soapg}
}

\newglossaryentry{soapg} {
	name={SOAP},
	description={SOAP ist ein Netzwerkprotokoll mit dem Information zwischen verschiedenen Computersystemen ausgetauscht und Remote Procedure Calls durchgeführt werden können. Dabei stützt sich SOAP bei den Nachrichten auf das XML Format. [siehe \url{http://.w3schools.com/soap/}]}
}

\newglossaryentry{WSDL} {
	type=\acronymtype,
	name={WSDL},
	description={Web Service Description Language},
	first={\textit{Web Service Description Language} (WSDL)},
	%see=[Glossar:]{wsdlg}
}

\newglossaryentry{wsdlg} {
	name={WSDL},
	description={WSDL ist eine Beschreibungssprache für Webservices. Sie enthält alle Methoden, die ein Webservice bereitstellt und zeigt wie diese zu implementieren sind.}
}

\newglossaryentry{Metadaten} {
	name={Metadaten},
	description={Metadaten sind beschreibende Daten. Sie enthalten Informationen und beschreibungen zu anderen Daten. Metadaten zum Datum ``Name'' sind zum Beispiel Informationen was dieses Datum aussagt und wie es zu verwenden ist.}
}

\newglossaryentry{jms} {
	type=\acronymtype,
	name={JMS},
	first={\textit{Java Message Service} (JMS)},
	description={Java Message Service},
	%see=[Glossar:]{jmsg}
}

\newglossaryentry{jmsg} {
	name={JMS},
	description={\textbf{Java Message Service} ist eine Programmierschnitstelle zum Senden und Empfangen von Nachrichten aus einem Client heraus. Ziel dabei ist es, lose gekoppelte asynchrone Kommunikation zwischen den Komponenten einer verteilten ANwendung zu ermöglichen.}
}

\newglossaryentry{simple}{
	name={Simple Framework},
	description={\href{http://simple.sourceforge.net/home.php}{Simple Framework} ist ein API in Java, mit der man eine \gls{xml}-Datenbindung erreichen kann. Es ist damit möglich, Daten aus einer XML-Schema-Instanz heraus automatisch an Java-Klassen zu binden, und diese Java-Klassen aus einem XML-Schema heraus zu generieren \cite{wiki:jaxb}. Damit können XML-Dokumente eingelesen, bearbeitet und wieder geschrieben werden. }
}

\newglossaryentry{https}{
	type=\acronymtype,
	name={HTTP/S},
	first={\textit{HyperText Transfer Protocol (Secure)} (HTTPS)},
	description={HyperText Transfer Protocol (Secure)},
	%see=[Glossar:]{httpsg}
}

\newglossaryentry{httpsg}{
	name={HyperText Transfer Protocol Secure},
	description={HyperText Transfer Protocol Secure ist ein  ein Kommunikationsprotokoll im World Wide Web, um Daten abhörsicher zu übertragen. Es stellt eine Transportverschlüsselung dar. \citation{https}}
}

\newglossaryentry{API}{
	type=\acronymtype,
	name={API},
	first={\textit{Application Programming Interface} (API)},
	description={Application Programming Interface},
	%see=[Glossar:]{APIg}
}

\newglossaryentry{APIg}{
	name={Application Programming Interface},
	description={Eine API ist eine Programmierschnittstelle, mit deren Hilfe es möglich ist, anderen Programmen eine Anbindung an das jeweilige System herzustellen. Hierzu existieren im Regelfall umfangreiche Dokumentationen für den Programmierer.}
}

\newglossaryentry{REST}{
	type=\acronymtype,
	name={REST},
	first={\textit{Representational State Transfer} (REST)},
	description={Representational State Transfer},
}



\newglossaryentry{URI}{
	type=\acronymtype,
	name={URI},
	first={\textit{Uniform Resource Identifier} (URI)},
	description={Uniform Resource Identifier},
}

\newglossaryentry{IMAP}{
	type=\acronymtype,
	name={IMAP},
	description={Internet Message Access Protocol},
	first={\textit{Internet Message Access Protocol} (IMAP)},
	%dsee=[Glossar:]{imap}
}

\newglossaryentry{POP}{
	type=\acronymtype,
	name={POP},
	description={Post Office Protocol},
	first={\textit{Post Office Protocol} (POP)},
	%dsee=[Glossar:]{pop}
}

\newglossaryentry{SMTP}{
	type=\acronymtype,
	name={SMTP},
	description={Simple Mail Transfer Protocol},
	first={\textit{Simple Mail Transfer Protocol} (SMTP)},
	%dsee=[Glossar:]{smtp}
}

\newglossaryentry{JAMES}{
	type=\acronymtype,
	name={JAMES},
	description={Java Apache Mail Enterprise Server},
	first={\textit{Java Apache Mail Enterprise Server} (Apache James)},
	%see=[Glossar:]{jamesg}
}

\newglossaryentry{jamesg}{
	name={Apache James},
	description={\gls{JAMES} ist ein Mailserver, der im Enterprise-Bereich eingesetzt wird. Er beruht dabei komplett auf der Programmiersprache Java, sodass die Plattformunabhängigkeit gewährleistet ist \cite{a:james}.}
}

\newglossaryentry{servlet}{
	name={Servlet},
	plural={Servlets},
	description={\gf{Servlets sind Java-Programme, die in einem besonders präparierten Java-Webserver ausgeführt werden.}\cite{rw:javainsel}} 
}
