% F�r umbr�che
\usepackage[ngerman]{babel}

% Umlaute
\usepackage[utf8]{inputenc}
\usepackage[T1]{fontenc}
\usepackage{lmodern}

% Es k�nnen Tabellen erstellt werden, die eine feste Gesamtbreite haben
\usepackage{tabularx}
\usepackage{tabulary}

\usepackage{layout}

% Euro-Zeichen etc
\usepackage{textcomp}

% Grafiken
\usepackage[dvips,final]{graphicx}
% Dort liegen die Bilder des Dokuments
\graphicspath{{images/}}
% Zum Umflie�en von Bildern
\usepackage{floatflt}

\usepackage[numbers, square]{natbib}

% Caption �ber der Tabelle / dem Code
\usepackage{floatrow}
\usepackage{tabu}

\usepackage{longtable}

\usepackage{dirtree}

\usepackage{fancyvrb}

%Definition der Farben in includes/colors
\usepackage[usenames,dvipsnames,table]{xcolor} 

% Package zum formatieren von Code
%\usepackage[chapter]{minted}

% Zum Einbinden von Programmcode
\usepackage{listings}

% Einstellungen f�r Listings, kann wahrscheinlich weg???
\lstset{%
    float=hbp,%
    basicstyle=\texttt\small, %
    identifierstyle=\color{colIdentifier}, %
    keywordstyle=\color{colKeys}, %
    stringstyle=\color{colString}, %
    commentstyle=\color{colComments}, %
    columns=flexible, %
    tabsize=2, %
    frame=single, %
    extendedchars=true, %
    showspaces=false, %
    showstringspaces=false, %
    numbers=left, %
    numberstyle=\tiny, %
    breaklines=true, %
    backgroundcolor=\color{hellgelb}, %
    breakautoindent=true, %
%    captionpos=b%
}


% Abk�rzungen
\usepackage{acronym}

% F�r Index-Ausgabe; \printindex
\usepackage{makeidx}

% Einfache Definition der Zeilenabst�nde und Seitenr�nder etc.
\usepackage{setspace}
\usepackage[top=3cm, bottom=3.5cm]{geometry}

\usepackage[
	automark,		% Kapitelangaben in Kopfzeile automatisch erstellen
	headsepline,	% Trennlinie unter Kopfzeile
	ilines			% Trennlinie linksb�ndig ausrichten
]{scrlayer-scrpage}

% PDF-Optionen
\usepackage[
	bookmarks,
	bookmarksopen=true,
	pdftitle={\titel},
	pdfauthor={\pdfautor},
	pdfcreator={\pdfautor},
	pdfsubject={\titel},
	pdfkeywords={\tags},
	colorlinks=true,
	linkcolor=magenta, 		% einfache interne Verkn�pfungen
	%linkcolor=black,
	anchorcolor=green,		% Ankertext
	citecolor=dhbwred,			% Verweise auf Literaturverzeichniseintr�ge im Text
	%citecolor=black,
	filecolor=magenta, 		% Lokale Dateien
	%filecolor=black,
	menucolor=compred,
	%menucolor=black, 		% Acrobat-Men�punkte
	urlcolor=cyan,
	%urlcolor=black, 
	plainpages=false,		% zur korrekten Erstellung der Bookmarks
	pdfpagelabels,			% zur korrekten Erstellung der Bookmarks
	hypertexnames=false,	% zur korrekten Erstellung der Bookmarks
	linktocpage 			% Seitenzahlen anstatt Text im Inhaltsverzeichnis verlinken
]{hyperref}

% URL 
\usepackage{url}

% G�nsef��chen mit \enquote{text}
\usepackage{csquotes}

% F�r die Chapter Formatierung 
\usepackage{blindtext, color}

% Zum fortlaufenden Durchnummerieren der Fu�noten
%\usepackage{chngcntr}

% Eigene Caption Definition mit Style
\usepackage{caption}

% Eigene Subcaption Definition mit Style
\usepackage{subcaption}

% Chapterstyle
\usepackage[explicit]{titlesec}
\usepackage{lmodern}

\usepackage{todonotes}

%https://en.wikibooks.org/wiki/LaTeX/Glossary
\usepackage[	 
	toc,   
	%xindy,
	%nomain,
	nopostdot,
	%Zahl, die Kapitel anzeigt, in dem das Wort gebraucht wurde
	nonumberlist,
	seeautonumberlist,
	acronym]      
{glossaries}

\usepackage{tikz}
\def\checkmark{\tikz\fill[scale=0.4](0,.35) -- (.25,0) -- (1,.7) -- (.25,.15) -- cycle;} 

\makeglossaries

\makeindex